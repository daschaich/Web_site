% ------------------------------------------------------------------
% For \input only
% Suppress indentation
\renewenvironment{spacelist} {
  \begin{list} {} {
    \setlength{\topsep}{-8 pt}
    \setlength{\itemsep}{5 pt}
    \setlength{\leftmargin}{-1.15 in}
  }
}{
  \end{list}
}

\raggedright
\vspace{12 pt}
\begin{spacelist}
  \item {\large \bfseries Monetary awards as principal investigator or a primary author of proposal} \\[10 pt]
    \begin{tabular}[t]{cl}
      2019 & \href{https://www.ukri.org}{UK Research and Innovation} Future Leader Fellowship {MR/S015418/1}, 943k GBP \\
           & \textit{New Frontiers of Lattice Field Theory}                                                            \\[6 pt]
      2018 & University of Bern Faculty of Science conference travel grant, 1700 CHF                                   \\[6 pt]
      2011 & National Science Foundation Award OISE-1107903, 5700 USD                                                  \\
           & \textit{Exploring the Origin of Mass with High-Performance Computing}                                     \\
    \end{tabular}

  \spacer
  \item {\large \bfseries Monetary awards as participating investigator} \\[10 pt]
    \begin{tabular}[t]{cl}
      2020 & \href{https://stfc.ukri.org}{Science and Technology Facilities Council} Consolidated Grant {ST/T000988/1}, 898k GBP \\
           & \textit{New Horizons in Quantum Field Theory, Particle Physics and String Phenomenology}                            \\[6 pt]
      2020 & Science and Technology Facilities Council Virtual Centre award, 19.5k GBP                                           \\
           & \textit{UK Lattice Field Theory Virtual Centre}                                                                     \\[6 pt]
      2020 & University of Liverpool Research Centre in Mathematics and Modelling workshop grant, 3000 GBP                       \\
           & \textit{Physics of strongly interacting fermions in low-dimensional materials}                                      \\
           & \hspace{2.4 cm} \textit{and high-energy physics: a numerical perspective} (cancelled due to COVID-19 pandemic)      \\
    \end{tabular}

  \vspace{18 pt}
  \item {\large \bfseries Monetary awards as project partner} \\[10 pt]
    \begin{tabular}[t]{cl}
      2020 & \href{https://epsrc.ukri.org}{Engineering and Physical Sciences Research Council} grant {EP/V001329/1}, 94k GBP \\
           &  Exascale Computing Algorithms and Infrastructures Benefitting UK Research (ExCALIBUR) programme                \\
           & \textit{Lattice Field Theory at the Exascale Frontier}                                                          \\
    \end{tabular}

  \vspace{18 pt}
  \item {\large \bfseries Computing allocations as principal investigator or a primary author of proposal} \\[10 pt]
    \begin{tabular}[t]{cl}
      2020 & \href{https://dirac.ac.uk}{DiRAC} proposal under review requesting 18M core hours                        \\ % Cambridge
           & \textit{Electroweak phenomenology from lattice strong dynamics}                                          \\[6 pt]
      2019 & \href{http://www.usqcd.org}{USQCD} computing allocation, 2.25M core hours                                \\ % Fermilab
           & \textit{Phases of a Higgs--Yukawa Theory and Symmetric Mass Generation}                                  \\[6 pt]
      2017 & USQCD computing allocation, 2.52M core hours                                                             \\ % Fermilab
           & \textit{Exploring Improved Methods to Extract the $0^{++}$ Mass in an SU(3) Gauge Theory with 8 Flavors} \\[6 pt]
      2017 & USQCD computing allocation, 12.5M core hours                                                             \\ % Fermilab
           & \textit{Thermodynamics of 3D Supersymmetric Yang--Mills}                                                 \\[6 pt]
      2017 & USQCD computing allocation, 2.46M core hours                                                             \\ % JLab
           & \textit{A new critical point in lattice four-fermion theories?}                                          \\[6 pt]
      2016 & \href{http://www.xsede.org}{XSEDE} computing allocation, 4.11M core hours                                \\ % Comet at SDSC
           & \textit{Lattice studies of supersymmetric gauge theories}                                                \\[6 pt]
      2016 & USQCD computing allocation, 11.37M core hours                                                            \\ % Fermilab
           & \textit{Lattice $\mathcal N = 4$ supersymmetric Yang--Mills on the Coulomb branch}                       \\[6 pt]
      2015 & USQCD computing allocation, 10.71M core hours                                                            \\ % Fermilab
           & \textit{Anomalous dimensions from lattice $\mathcal N = 4$ super Yang--Mills with an improved action}    \\[6 pt]
      2014 & USQCD computing allocation, 11.04M core hours                                                            \\ % Fermilab
           & \textit{Lattice $\mathcal N = 4$ supersymmetric Yang--Mills with 2, 3 and 4 colors}                      \\[6 pt]
      2013 & \href{https://www.top500.org/system/176922}{Janus} computing allocation, 4.8M core hours                 \\ % University of Colorado
           & \textit{Lattice studies of an infrared-conformal gauge theory}                                           \\[6 pt]
      2013 & USQCD computing allocation, 9.97M core hours                                                             \\ % Fermilab
           & \textit{Eight-flavor SU(3) gauge theory with nHYP-smeared fermions}                                      \\[6 pt]
    \end{tabular} % TODO: Page break hack
    \begin{tabular}[t]{cl}
      2013 & XSEDE computing allocation, 7.0M core hours                                                              \\ % 3.5 on Gordon at SDSC, 3.5 on Stampede at TACC
           & \textit{SU(3) gauge theories with many fermions --- to the chiral limit}                                 \\[6 pt]
      2012 & Janus computing allocation, 5.5M core hours                                                              \\ % University of Colorado
           & \textit{Lattice studies of strongly-interacting gauge theories with many light fermions}                 \\[6 pt]
      2012 & USQCD computing allocation, 4.84M core hours                                                             \\ % Fermilab
           & \textit{Many flavor gauge theories: finite volume scaling at small masses}                               \\[6 pt]
      2011 & XSEDE computing allocation, 2.5M core hours                                                              \\ % Gordon at SDSC
           & \textit{Phase structure of SU(3) gauge theory with many light fermions}                                  \\
    \end{tabular}

  \vspace{18 pt}
  \item {\large \bfseries Computing allocations as participating investigator} \\[10 pt]
    \begin{tabular}[t]{cl}
      2020 & USQCD computing allocation, 13.2M core hours                                                         \\ % JLab (KNL)
           & \textit{Composite Higgs model with four light and six heavy flavors}                                 \\[6 pt]
      2020 & USQCD computing allocation, 1.07M core hours                                                         \\ % Fermilab
           & \textit{Decoupling doublers using generalized Yukawa interactions}                                   \\[6 pt]
      2019 & USQCD computing allocation, 8.75M core hours                                                         \\ % BNL (KNL)
           & \textit{Composite Higgs model with four light and six heavy flavors}                                 \\[6 pt]
      2018 & USQCD computing allocation, 12M core hours                                                           \\ % Fermilab
           & \textit{Thermodynamics of SYM theory in three and four dimensions}                                   \\[6 pt]
      2018 & USQCD computing allocation, 29M core hours                                                           \\ % BNL (KNL)
           & \textit{Composite Higgs model with four light and six heavy flavors}                                 \\[6 pt]
      2017 & USQCD computing allocation, 12M core hours                                                           \\ % JLab (KNL)
           & \textit{Simulations of four light and six heavy flavors using smeared M\"obius domain-wall fermions} \\[6 pt]
      2016 & USQCD computing allocation, 9.51M core hours                                                         \\ % Fermilab
           & \textit{Measuring the Low Energy Effective Theory in Multiflavor QCD}                                \\[6 pt]
      2016 & ASCR Leadership Computing Challenge allocation, 55M core hours                                       \\ % Mira at Argonne
           & \textit{Exploring Higgs Compositeness Mechanism in the Era of the 14 TeV LHC}                        \\[6 pt]
      2015 & USQCD computing allocation, 8.6M core hours                                                          \\ % Fermilab
           & \textit{Non-Perturbative Collider Phenomenology of Stealth Dark Matter}                              \\[6 pt]
      2014 & USQCD computing allocation, 13.33M core hours                                                        \\ % Fermilab
           & \textit{Electromagnetic Polarizability of Bosonic Composite Dark Matter}                             \\[6 pt]
% Temporarily suppressed to fit onto two pages...
%      2013 & USQCD computing allocation, 29k GPU hours                                                            \\ % Fermilab
%           & \textit{$\mathcal N = 4$ Super Yang--Mills on GPUs}                                                  \\[6 pt]
      2013 & USQCD computing allocation, 9.71M core hours                                                         \\ % Fermilab
           & \textit{Lattice study of $\mathcal N = 4$ Super Yang--Mills}                                         \\[6 pt]
      2013 & USQCD computing allocation, 9.35M core hours                                                         \\ % Fermilab
           & \textit{Two-Color Gauge Theories in the Higgs Era}                                                   \\[6 pt]
      2013 & Janus computing allocation, 1.6M core hours                                                          \\ % University of Colorado
           & \textit{Finite size scaling studies with twelve light fermions}                                      \\[6 pt]
      2013 & XSEDE computing allocation, 9.2M core hours                                                          \\ % Kraken
           & \textit{Many-Fermion Gauge Theories for TeV Physics}                                                 \\[6 pt]
      2012 & USQCD computing allocation, 12.4M core hours                                                         \\ % Fermilab
           & \textit{Extended study of many fermion gauge theories for TeV physics}                               \\[6 pt]
      2012 & USQCD computing allocation, 86.3k GPU hours                                                          \\ % JLab
           & \textit{Disconnected contributions to nucleon form factors with chiral fermions}                     \\[6 pt]
      2011 & USQCD computing allocation, 12.08M core hours                                                        \\ % JLab
           & \textit{Exploration of Many-Fermion Gauge Theories for TeV Physics}                                  \\[6 pt]
      2010 & USQCD computing allocation, 5M core hours                                                            \\ % JLab(?)
           & \textit{Two-Color Gauge Theories for TeV Physics}                                                    \\[6 pt]
      2009 & USQCD computing allocation, 580k core hours                                                          \\ % Fermilab
           & \textit{Strange quark contribution to nucleon form factors}                                          \\[6 pt]
      2008 & USQCD computing allocation, 488k core hours                                                          \\ % Fermilab
           & \textit{QCD Vacuum Polarization Contribution to the S Parameter and g-2}                             \\
    \end{tabular}
\end{spacelist}
% ------------------------------------------------------------------
