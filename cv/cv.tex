% ------------------------------------------------------------------
% Based on template provided by Theodore Pavlic
% http://www.tedpavlic.com/post_resume_cv_latex_example.php
\documentclass[10 pt]{article}
\usepackage{calc}
\usepackage{revnum}

% Put the section titles on left side of page
\reversemarginpar

% Page geometry -- marginparwidth sets length of section titles,
%                  while marginparsep sets space between titles and text
%\usepackage[paper=letterpaper, marginparwidth=1 in, marginparsep=0.15 in, margin=0.75 in, includemp]{geometry}
\usepackage[paper=a4paper, marginparwidth=1 in, marginparsep=0.15 in, margin=0.62 in, includemp]{geometry}

\usepackage[none]{hyphenat}
\sloppy

% Fancy page layout
\usepackage{fancyhdr, lastpage}
\pagestyle{fancy}
%\pagestyle{empty}    % Uncomment to get rid of page numbers in footer

% Header (title) with a horizontal rule under it
% Use Large instead of large due to 10-point base style
\setlength{\parindent}{0 in}    % Needed for header, could overwrite below
\fancyhf{}\renewcommand{\headrulewidth}{0pt}
\newcommand{\makeheading}[1]%
    {\hspace*{-\marginparsep minus \marginparwidth}%
     \begin{minipage}[t]{\textwidth+\marginparwidth+\marginparsep}%
        {\Large \bfseries #1}\\[-0.15\baselineskip]%
         \rule{\columnwidth}{1pt}%
     \end{minipage}}

% Footnote formatting
\fancyfootoffset{\marginparsep+\marginparwidth}
\newlength{\footpageshift}
\setlength{\footpageshift}
      {0.5\textwidth+0.5\marginparsep+0.5\marginparwidth-2in}
\lfoot{\hspace{\footpageshift}%
       \parbox{4in}{\, \hfill %
       CV page \arabic{page} of \protect\pageref*{LastPage} % +LP
%       \arabic{page}                 % -LP
       \hfill \,}}

% Link formatting
\usepackage{color, hyperref}
\definecolor{darkblue}{rgb}{0.0,0.0,0.5}
\hypersetup{colorlinks, breaklinks, linkcolor=darkblue, urlcolor=darkblue, anchorcolor=darkblue, citecolor=darkblue, pdfstartview={FitH}, pdfnewwindow=true, pdfpagemode=UseNone}

% Section headings
\renewcommand{\section}[2]%
    {\pagebreak[2]\vspace{1.3\baselineskip}%
%     \phantomsection\addcontentsline{toc}{section}{#1}% % DS
     \hspace{0in}%
     \marginpar{
     \raggedright \scshape #1}#2}

% A basic list with no indentation or space between the lines
\newenvironment{tightlist}
  {\begin{list} {} {\setlength{\topsep}{-8 pt} \setlength{\itemsep}{-3 pt} \setlength{\leftmargin}{0 mm}}}{\end{list}}

% A basic list with no indentation and a little space between the lines
\newenvironment{spacelist}
  {\begin{list} {} {\setlength{\topsep}{-8 pt} \setlength{\itemsep}{5 pt} \setlength{\leftmargin}{0 mm}}}{\end{list}}

% To add some paragraph space between lines
% This also tells LaTeX to preferably break a page on one of these gaps if there is a needed pagebreak nearby
\newcommand{\blankline}{\quad\pagebreak[2]}
\newcommand{\spacer}{\blankline\vspace{12 pt}\blankline}

% Try to ensure that pagebreaks don't split up items
\newcommand{\pagebreakitem}{\item\pagebreak[2]}

% Turn italicization of titles on or off
\newcommand{\optem}[1]{\textit{#1}}
%\newcommand{\optem}[1]{``#1''}

% Standard shortcuts
\usepackage{verbatim} % For comment environment
\newcommand{\lra}{\ensuremath{\longrightarrow} }
% ------------------------------------------------------------------



% ------------------------------------------------------------------
% The actual content
\begin{document}
\raggedright
\makeheading{David Schaich --- Curriculum vitae \hfill \textnormal{\footnotesize{30 June 2019}}}
% ------------------------------------------------------------------



% ------------------------------------------------------------------
\section{Contact}
\begin{tabular}[t]{ll}
  Department of Mathematical Sciences \hspace{1 cm} & +44 (0)151 794 3778 (Office)                                               \\
  University of Liverpool                           & +44 (0)7568 168895 (Mobile)                                                \\
  Liverpool L69 7ZL                                 & Skype: daschaich                                                           \\ % Ekiga: schaich
  United Kingdom                                    & \href{mailto:david.schaich@liverpool.ac.uk}{david.schaich@liverpool.ac.uk} \\
  \url{http://www.davidschaich.net}                 &                                                                            \\
\end{tabular}

\spacer
% ------------------------------------------------------------------



% ------------------------------------------------------------------
\section{Employment} % The next line must be blank!

\vspace{-12 pt} % Get things to line up
\begin{tabular}[t]{cl}
  2019--present & Lecturer, University of Liverpool                               \\
  2016--2019    & Postdoctoral Researcher, University of Bern             \\
  2013--2016    & Postdoctoral Researcher, Syracuse University            \\
  2011--2013    & Postdoctoral Researcher, University of Colorado Boulder \\
\end{tabular}

\spacer
% ------------------------------------------------------------------



% ------------------------------------------------------------------
\section{Education} % The next line must be blank!

\vspace{-12 pt} % Get things to line up
\begin{tabular}[t]{cl}
  2011  & Ph.D., Physics, Boston University                                                 \\
  2011  & Certificate in Computational Science, Boston University                           \\
  2008  & M.A., Physics, Boston University                                                  \\
  2006  & B.A. \textit{summa cum laude}, Physics, History, and Mathematics, Amherst College \\
\end{tabular}

\spacer
% ------------------------------------------------------------------



% ------------------------------------------------------------------
\section{Publication metrics} % The next line must be blank!

\vspace{-12 pt} % Get things to line up
57 papers; \ 1455 citations; \ \href{https://en.wikipedia.org/wiki/H-index}{$h = 23$}; \ \href{https://en.wikipedia.org/wiki/G-index}{$g = 36$} \\[1 pt]
See also \href{http://inspirehep.net/author/profile/D.Schaich.1}{inspirehep.net/author/profile/D.Schaich.1}

\spacer
% ------------------------------------------------------------------



% ------------------------------------------------------------------
\section{Recent grants} % The next line must be blank!

\vspace{-12 pt} % Get things to line up
\begin{tabular}[t]{cl}
  2019 & \href{http://www.usqcd.org}{USQCD} computing allocation, 10M core hours                                                                                                                                                   \\ % BNL (KNL)
       & \textit{Composite Higgs model with four light and six heavy flavors}                                                                                                                                                      \\[6 pt]
  2019 & USQCD computing allocation, 2.25M core hours                                                                                                                                                                              \\ % Fermilab
       & \textit{Phases of a Higgs--Yukawa Theory and Symmetric Mass Generation}                                                                                                                                                   \\[6 pt]
  2019 & \href{https://www.ukri.org/funding/funding-opportunities/future-leaders-fellowships/meet-our-future-leaders-fellows/david-schaich-university-of-liverpool/}{UK Research \& Innovation Future Leader Fellowship}, 923k GBP \\
       & \textit{New Frontiers of Lattice Field Theory}                                                                                                                                                                            \\[6 pt]
  2018 & University of Bern Faculty of Science conference travel grant, 1700 CHF                                                                                                                                                   \\[6 pt]
  2018 & USQCD computing allocation, 29M core hours                                                                                                                                                                                \\ % BNL (KNL)
       & \textit{Composite Higgs model with four light and six heavy flavors}                                                                                                                                                      \\[6 pt]
  2018 & USQCD computing allocation, 12M core hours                                                                                                                                                                                \\ % Fermilab
       & \textit{Thermodynamics of SYM theory in three and four dimensions}                                                                                                                                                        \\[6 pt]
  2017 & USQCD computing allocation, 2.52M core hours                                                                                                                                                                              \\ % Fermilab
       & \textit{Exploring Improved Methods to Extract the 0$^{++}$ Mass} \\ & \textit{in an SU(3) Gauge Theory with 8 Flavors}                                                                                                    \\[6 pt]
  2017 & USQCD computing allocation, 12.5M core hours                                                                                                                                                                              \\ % Fermilab
       & \textit{Thermodynamics of 3D Supersymmetric Yang--Mills}                                                                                                                                                                  \\[6 pt]
  2017 & USQCD computing allocation, 2.46M core hours                                                                                                                                                                              \\ % JLab
       & \textit{A new critical point in lattice four-fermion theories?}                                                                                                                                                                                                                                                              \\
\end{tabular}

\begin{comment}
\blankline % !!! Less vertical space

\section{Pending grants} % The next line must be blank!

\vspace{-12 pt} % Get things to line up
\begin{tabular}[t]{cl}
  2019 & \textit{Phases of a Higgs--Yukawa Theory and Symmetric Mass Generation} \\ % Fermilab
       & (2.25M core-hour request submitted to USQCD)                            \\[6 pt]
  2019 & \textit{Composite Higgs model with four light and six heavy flavors}    \\ % BNL (KNL)
       & (8.75M core-hour request submitted to USQCD)                            \\
\end{tabular}
\end{comment}

\spacer
% ------------------------------------------------------------------



% ------------------------------------------------------------------
\section{Honors and awards} % The next line must be blank!

\vspace{-12 pt} % Get things to line up
\begin{tabular}[t]{cl}
  2019--2026  & UK Research \& Innovation Future Leader Fellowship      \\
  2007--2009  & US National Science Foundation IGERT Fellowship         \\
     2008     & Boston University Physics Department Chair's Book Prize \\
  2006--2007  & Boston University Dean's Fellowship                     \\
  2006--2007  & Forris Jewett Moore Fellowship from Amherst College     \\
     2006     & Sigma Xi, The Scientific Research Society               \\
     2005     & Phi Beta Kappa                                          \\
\end{tabular}

\newpage
% ------------------------------------------------------------------



% ------------------------------------------------------------------
\section{Short-term appointments} % The next line must be blank!

\vspace{-12 pt} % Get things to line up
\begin{tabular}[t]{cl}
  2018  & International Centre for Theoretical Sciences, Bangalore, January--February \\
  2016  & Kavli Institute for Theoretical Physics, Santa Barbara, February--March     \\
  2015  & Humboldt University, Berlin, November--December                             \\
  2015  & Kavli Institute for Theoretical Physics, Santa Barbara, August--September   \\
  2015  & Aspen Center for Physics, May--June                                         \\
  2013  & Aspen Center for Physics, May--June                                         \\
  2011  & National Center for Theoretical Sciences, Taipei, June--August              \\
  2010  & Lawrence Livermore National Lab, January--February                          \\
  2005  & CERN, June--August                                                          \\
%  2003  & Hope College, May--July                                                     \\
\end{tabular}

\spacer
% ------------------------------------------------------------------



% ------------------------------------------------------------------
\section{Teaching} % The next line must be blank!

\vspace{-12 pt} % Get things to line up
\begin{tabular}[t]{cl}
  Fall 2017  & \href{http://www.davidschaich.net/teaching/1718F_BSM/index.html}{New strong dynamics beyond the standard model} (graduate), U.~Bern      \\
  2013--2014 & \href{http://www.davidschaich.net/teaching/1314_AdQFT/index.html}{Advanced quantum field theory} (graduate, informal), Syracuse U. \\
  2012--2013 & \href{http://www.davidschaich.net/teaching/1213_AdQFT/index.html}{Advanced Quantum Field Theory} (graduate, informal), U.~Colorado \\
\end{tabular}

\spacer
% ------------------------------------------------------------------



% ------------------------------------------------------------------
\section{Workshop organization} % The next line must be blank!

\vspace{-12 pt} % Get things to line up
\begin{tabular}[t]{cl}
  2018  & \href{http://www.ectstar.eu/node/4226}{Interdisciplinary approach to QCD-like composite dark matter}                                                                        \\
        & ECT* Trento, 1--5 October                                                                                                                                                   \\[6 pt]
  2018  & \href{https://higgs.ph.ed.ac.uk/workshops/numerical-approaches-holography-quantum-gravity-and-cosmology}{Numerical approaches to holography, quantum gravity and cosmology} \\
        & Higgs Centre for Theoretical Physics, University of Edinburgh, 21--24 May                                                                                                   \\[6 pt]
  2014  & \href{http://blogs.bu.edu/ppcm/}{Field Theoretic Computer Simulations for Particle Physics and Condensed Matter}                                                            \\
        & Boston University, 8--10 May                                                                                                                                                \\[6 pt]
  2012  & \href{http://www-hep.colorado.edu/~schaich/lat-exp-2012/}{Lattice Meets Experiment: Beyond the Standard Model}, U.~Colorado, 26--27 October                                 \\[6 pt]
  2010  & \href{http://www-hep.colorado.edu/~schaich/QCDNA6/}{QCDNA6: Numerical Analysis for Lattice Gauge Theory}, Boston U., 8--10 September                                        \\[6 pt]
  2009  & \href{http://www-hep.colorado.edu/~schaich/LGTforLHC2009/}{Lattice Gauge Theory for LHC Physics}, Boston University, 6--7 November                                          \\
\end{tabular}

\spacer
% ------------------------------------------------------------------



% ------------------------------------------------------------------
\section{Undergrad supervision} % The next line must be blank!

\vspace{-12 pt} % Get things to line up
\begin{tabular}[t]{cl}
  2019 & Harry Lee, \textit{Quantum Computing} (second reader, advisor: Paul Rakow) \\
\end{tabular}

\spacer
% ------------------------------------------------------------------




% ------------------------------------------------------------------
\section{PhD students mentored} % The next line must be blank!

\vspace{-12 pt} % Get things to line up
\begin{tabular}[t]{cl}
  2014--2019    & Raghav Jha (advisor: Simon Catterall), Syracuse University            \\
                & \qquad \lra Postdoc (Perimeter)                                       \\[6 pt]
  2013--2015    & Aarti Veernala (advisor: Simon Catterall), Syracuse University        \\
                & \qquad \lra Postdoc (Fermilab)                                        \\[6 pt]
  2011--2014    & Gregory Petropoulos (advisor: Anna Hasenfratz), U.~Colorado Boulder   \\
                & \qquad \lra Data science industry (CenturyLink Cognilytics)           \\[6 pt]
  2011--2014    & Anqi Cheng (advisor: Anna Hasenfratz), University of Colorado Boulder \\
                & \qquad \lra Data science industry (Rule14)                            \\
\end{tabular}

\spacer
% ------------------------------------------------------------------



% ------------------------------------------------------------------
\section{Journal referee} % The next line must be blank!

\vspace{-12 pt} % Get things to line up
\begin{tightlist} % publons.com/author/1444034/
%  \item Physical Review Letters, Physical Review D, Journal of High Energy Physics, Physical Review B, Nuclear Physics B, International Journal of Modern Physics A
  \item Physical Review Letters, 7 reviews since 2014
  \item Physical Review D, 5 reviews since 2012
  \item Journal of High Energy Physics, 2 reviews since 2017
  \item Physical Review B, 4 reviews since 2012
  \item Nuclear Physics B, 1 review since 2014
  \item International Journal of Modern Physics A, 2 reviews since 2017
\end{tightlist}

\newpage
% ------------------------------------------------------------------



% ------------------------------------------------------------------
\section{Other professional service} % The next line must be blank!

\vspace{-12 pt} % Get things to line up
\begin{tightlist}
  \item Parallel session chair:                                                                                                                \\
    \qquad \href{https://web.pa.msu.edu/conf/Lattice2018/}{Lattice 2018} (Physics Beyond the Standard Model)                                   \\
    \qquad \href{http://wpd.ugr.es/~lattice2017/}{Lattice 2017} (Applications Beyond QCD)                                                      \\
  \item Organizer, High Energy Theory and Cosmology Seminars, Syracuse U., 2013--2016                                                          \\
  \item Organizer, High Energy Theory Journal Club, U.~Colorado, 2012--2013                                                                    \\
  \item Member, USQCD Software Committee, 2013--2016                                                                                           \\
  \item Contributor to \href{http://www.physics.utah.edu/~detar/milc/}{MIMD Lattice Computation (MILC)} software, 2012
  \item Amherst College \href{https://www.amherst.edu/campuslife/careers/mentoring}{alumnus mentor}, seven undergraduates mentored, 2013--2017 \\ % Six long-term, one quick
%  \item \href{http://www.ectstar.eu}{ECT* Trento} Associate, since 2018                                                                        \\ % Haven't yet done anything for this...
%  \item Ally Physicist on \href{http://lgbtphysicists.org}{lgbtphysicists.org}, since 2016                                                     \\ % Haven't really done anything for this...
\end{tightlist}

\spacer
% ------------------------------------------------------------------



% ------------------------------------------------------------------
\section{Additional training} % The next line must be blank!

\vspace{-12 pt} % Get things to line up
\begin{tabular}[t]{cl}
  2016  & Advancing Learning through Evidence-Based STEM Teaching, 1 June--31 July         \\ %Center for the Integration of Research, Teaching, and Learning
  2015  & An Introduction to Evidence-Based Undergraduate STEM Teaching, 28 Sept.--19 Nov. \\ %Center for the Integration of Research, Teaching, and Learning
  2009  & Les Houches Summer School in Lattice Gauge Theory, 3--28 August                  \\ %Les Houches, France
  2007  & CTEQ Summer School on QCD Analysis and Phenomenology, 30 May--7 June             \\ %Madison WI
  2005  & CERN Summer Students Programme, 6 July--12 August                                \\ %Geneva, Switzerland
\end{tabular}

\spacer
% ------------------------------------------------------------------



% ------------------------------------------------------------------
\section{Professional memberships} % The next line must be blank!

\vspace{-12 pt} % Get things to line up
\begin{tightlist}
  \item American Physical Society, since 2006
  \item European Physical Society, since 2018
  \item Institute of Physics, since 2018
  \item Institute of Mathematics and its Applications, since 2019
  \item American Association for the Advancement of Science, since 2013
  \item Sigma Xi, The Scientific Research Society, since 2006
  \item Free Software Foundation, since 2010
  \item Swiss Institute of Particle Physics, 2016--2019
\end{tightlist}

\spacer
% ------------------------------------------------------------------



% ------------------------------------------------------------------
\section{Research} % The next line must be blank!

\vspace{-12 pt} % Get things to line up
\begin{tightlist}
  \item \textbf{Application of high-performance computing to particle physics}
  \item ---Investigations of strongly interacting quantum field theories using lattice gauge theory
  \item ---Non-perturbative lattice studies of supersymmetric field theories
  \item ---Quantum chromodynamics at non-zero baryon density
  \item ---Testing models of dynamical electroweak symmetry breaking and composite dark matter
\end{tightlist}

\spacer
% ------------------------------------------------------------------



% ------------------------------------------------------------------
\section{Technical expertise} % The next line must be blank!

\vspace{-12 pt} % Get things to line up
\begin{tightlist}
  \item \textbf{Programming:} C; \ Python; \ Bash; \ Fortran; \ Perl
  \item \textbf{High-performance computing:} MPI-based lattice domain-specific languages (MILC, QDP)
  \item \textbf{Version control:} git; svn
  \item \textbf{System administration:} GNU/Linux
  \item \textbf{Markup \& Web:} \LaTeX{}; \ B\textsc{ib}\TeX{}; \ HTML; \ PHP; \ SQL
\end{tightlist}

\spacer
% ------------------------------------------------------------------



% ------------------------------------------------------------------
\section{Languages} % The next line must be blank!

\vspace{-12 pt} % Get things to line up
English (native); \ Spanish (intermediate/B1); \ German (elementary/A2); \ Welsh (basic/A1)
% ------------------------------------------------------------------



% ------------------------------------------------------------------
\newpage
%\makeheading{Publications \hfill \textnormal{{\footnotesize (1455 citations; \ \href{https://en.wikipedia.org/wiki/H-index}{$h = 23$}; \ \href{https://en.wikipedia.org/wiki/G-index}{$g = 36$}; \ \href{http://inspirehep.net/author/profile/D.Schaich.1}{inspirehep.net/author/profile/D.Schaich.1})}}}
\makeheading{Publications \hfill \textnormal{{\footnotesize (Convention: authors in alphabetical order)}}} \\[24 pt]
% ------------------------------------------------------------------
% For \input only
% Suppress indentation
\renewenvironment{spacelist} {
  \begin{list} {} {
    \setlength{\topsep}{-8 pt}
    \setlength{\itemsep}{5 pt}
    \setlength{\leftmargin}{-1.15 in}
  }
}{
  \end{list}
}

\raggedright

\begin{spacelist}
  \item {\large \bfseries Citation metrics} \\[6 pt]
  1682 citations; \ \ \href{https://en.wikipedia.org/wiki/H-index}{$h = 24$}; \ \ \href{https://en.wikipedia.org/wiki/G-index}{$g = 39$} \\[2 pt]
  See also \href{http://inspirehep.net/author/profile/D.Schaich.1}{inspirehep.net/author/profile/D.Schaich.1} \\[15 pt]

  \item {\large \bfseries Refereed journal articles}
  \begin{revnumerate}
    \setlength{\topsep}{-8 pt}
    \setlength{\itemsep}{10 pt}
    \setlength{\leftmargin}{0 mm}
    \pagebreakitem
      \optem{Near-conformal dynamics in a chirally broken system} \\
      LSD Collaboration: Thomas Appelquist, Richard C.~Brower, Kimmy K.~Cushman, George T.~Fleming \textit{et al.} \\ % Andrew D.~Gasbarro, Anna Hasenfratz, Xiao-Yong Jin, Ethan T.~Neil, James C.~Osborn, Claudio Rebbi, Enrico Rinaldi, David Schaich, Pavlos Vranas and Oliver Witzel
      Submitted to \textit{Physical Review Letters} (2020) [\href{http://arxiv.org/abs/2007.01810}{arXiv:2007.01810}] \\
    \pagebreakitem
      \optem{Nonpeturbative investigations of SU(3) gauge theory with eight dynamical flavors} \\
      LSD Collaboration: Richard C.~Brower, Kimmy Cushman, George T.~Fleming, Andrew Gasbarro \textit{et al.} \\ % Anna Hasenfratz, Xiao-Yong Jin, Graham D.~Kribs, Ethan T.~Neil, James C.~Osborn, Claudio Rebbi, Enrico Rinaldi, David Schaich, Pavlos Vranas and Oliver Witzel
      Submitted to \textit{Physical Review D} (2020) [\href{http://arxiv.org/abs/2006.16429}{arXiv:2006.16429}] \\
      Data release at \href{https://doi.org/10.5281/zenodo.3921870}{doi:10.5281/zenodo.3921870}
    \pagebreakitem
      \optem{Nonpeturbative investigations of SU(3) gauge theory with eight dynamical flavors} \\
      LSD Collaboration: Thomas Appelquist, Richard C.~Brower, George T.~Fleming, Andrew Gasbarro \textit{et al.} \\ % Anna Hasenfratz, Xiao-Yong Jin, Ethan T.~Neil, James C.~Osborn, Claudio Rebbi, Enrico Rinaldi, David Schaich, Pavlos Vranas, Evan Weinberg and Oliver Witzel
      \href{http://dx.doi.org/10.1103/PhysRevD.99.014509}{\textit{Physical Review D} \textbf{99}:014509} (2019) [\href{http://arxiv.org/abs/1807.08411}{arXiv:1807.08411}]
    \pagebreakitem
      \optem{SO(4) invariant Higgs--Yukawa model with reduced staggered fermions} \\
      Nouman Butt, Simon Catterall and David Schaich \\
      \href{http://dx.doi.org/10.1103/PhysRevD.98.114514}{\textit{Physical Review D} \textbf{98}:114514} (2018) [\href{http://arxiv.org/abs/1810.06117}{arXiv:1810.06117}]
    \pagebreakitem
      \optem{Linear Sigma EFT for Nearly Conformal Gauge Theories} \\
      LSD Collaboration: Thomas Appelquist, Richard C.~Brower, George T.~Fleming, Andrew Gasbarro \textit{et al.} \\ % Anna Hasenfratz, James Ingoldby, Joe Kiskis, James C.~Osborn, Claudio Rebbi, Enrico Rinaldi, David Schaich, Pavlos Vranas, Evan Weinberg and Oliver Witzel
      \href{http://dx.doi.org/10.1103/PhysRevD.98.114510}{\textit{Physical Review D} \textbf{98}:114510} (2018, Editors' Suggestion) [\href{http://arxiv.org/abs/1809.02624}{arXiv:1809.02624}]
    \pagebreakitem
      \optem{Solution of the sign problem in the Potts model at fixed fermion number} \\
      Andrei Alexandru, Georg Bergner, David Schaich and Urs Wenger \\
      \href{http://dx.doi.org/10.1103/PhysRevD.97.114503}{\textit{Physical Review D} \textbf{97}:114503} (2018) [\href{http://arxiv.org/abs/1712.07585}{arXiv:1712.07585}]
    \pagebreakitem
      \optem{Testing holography using lattice super-Yang--Mills on a 2-torus} \\
      Simon Catterall, Raghav G.~Jha, David Schaich and Toby Wiseman \\
      \href{http://dx.doi.org/10.1103/PhysRevD.97.086020}{\textit{Physical Review D} \textbf{97}:086020} (2018) [\href{http://arxiv.org/abs/1709.07025}{arXiv:1709.07025}]
    \pagebreakitem
      \optem{Nonperturbative $\beta$ function of twelve-flavor SU(3) gauge theory} \\
      Anna Hasenfratz and David Schaich \\
      \href{http://dx.doi.org/10.1007/JHEP02(2018)132}{\textit{Journal of High Energy Physics} \textbf{1802}:132} (2018) [\href{http://arxiv.org/abs/1610.10004}{arXiv:1610.10004}]
    \pagebreakitem
      \optem{Novel phases in strongly coupled four-fermion theories} \\
      Simon Catterall and David Schaich \\
      \href{http://dx.doi.org/10.1103/PhysRevD.96.034506}{\textit{Physical Review D} \textbf{96}:034506} (2017) [\href{http://arxiv.org/abs/1609.08541}{arXiv:1609.08541}]
    \pagebreakitem
      \optem{Strongly interacting dynamics and the search for new physics at the LHC} \\
      LSD Collaboration: Thomas Appelquist, Richard C.~Brower, George T.~Fleming, Anna Hasenfratz \textit{et al.} \\ % Xiao-Yong Jin, Joe Kiskis, Ethan T. Neil, James C. Osborn, Claudio Rebbi, Enrico Rinaldi, David~Schaich, Pavlos Vranas, Evan Weinberg and Oliver Witzel
      \href{http://dx.doi.org/10.1103/PhysRevD.93.114514}{\textit{Physical Review D} \textbf{93}:114514} (2016) [\href{http://arxiv.org/abs/1601.04027}{arXiv:1601.04027}]
    \pagebreakitem
      \optem{Detecting Stealth Dark Matter Directly through Electromagnetic Polarizability} \\
      LSD Collaboration: Thomas Appelquist, Evan Berkowitz, Richard C.~Brower, Michael I.~Buchoff \textit{et al.} \\ %, George T.~Fleming, Xiao-Yong Jin, Joe Kiskis, Graham D.~Kribs, Ethan T.~Neil, James~C.~Osborn, Claudio Rebbi, Enrico Rinaldi, David~Schaich, Chris Schroeder, Sergey Syritsyn, Pavlos Vranas, Evan Weinberg and Oliver Witzel
      \href{http://dx.doi.org/10.1103/PhysRevLett.115.171803}{\textit{Physical Review Letters} \textbf{115}:171803} (2015, Editors' Suggestion) [\href{http://arxiv.org/abs/1503.04205}{arXiv:1503.04205}]
    \pagebreakitem
      \optem{Stealth dark matter: Dark scalar baryons through the Higgs portal} \\
      LSD Collaboration: Thomas Appelquist, Richard C.~Brower, Michael I.~Buchoff, George T.~Fleming \textit{et al.} \\ %, Xiao-Yong Jin, Joe Kiskis, Graham D.~Kribs, Ethan T.~Neil, James~C.~Osborn, Claudio Rebbi, Enrico Rinaldi, David~Schaich, Chris Schroeder, Sergey Syritsyn, Pavlos Vranas, Evan Weinberg and Oliver Witzel
      \href{http://dx.doi.org/10.1103/PhysRevD.92.075030}{\textit{Physical Review D} \textbf{92}:075030} (2015, Editors' Suggestion) [\href{http://arxiv.org/abs/1503.04203}{arXiv:1503.04203}]
    \pagebreakitem
      \optem{Lifting flat directions in lattice supersymmetry} \\
      Simon Catterall and David Schaich \\
      \href{http://dx.doi.org/10.1007/JHEP07(2015)057}{\textit{Journal of High Energy Physics} \textbf{1507}:057} (2015) [\href{http://arxiv.org/abs/1505.03135}{arXiv:1505.03135}]
    \pagebreakitem
      \optem{Nonperturbative $\beta$ function of eight-flavor SU(3) gauge theory} \\
      Anna Hasenfratz, David Schaich and Aarti Veernala \\
      \href{http://dx.doi.org/10.1007/JHEP06(2015)143}{\textit{Journal of High Energy Physics} \textbf{1506}:143} (2015) [\href{http://arxiv.org/abs/1410.5886}{arXiv:1410.5886}]
    \pagebreakitem
      \optem{Parallel software for lattice $\mathcal N = 4$ supersymmetric Yang--Mills theory} \\
      David Schaich and Thomas DeGrand \\
      \href{http://dx.doi.org/10.1016/j.cpc.2014.12.025}{\textit{Computer Physics Communications} \textbf{190}:200--212} (2015) [\href{http://arxiv.org/abs/1410.6971}{arXiv:1410.6971}]
    \pagebreakitem
      \optem{Lattice simulations with eight flavors of domain wall fermions in SU(3) gauge theory} \\
      LSD Collaboration: Thomas Appelquist, Richard C.~Brower, George T.~Fleming, Joe Kiskis, Meifeng Lin \textit{et al.} \\ %, Ethan T.~Neil, James~C.~Osborn, Claudio Rebbi, Enrico Rinaldi, David~Schaich, Chris Schroeder, Sergey Syritsyn, Gennady~Voronov, Pavlos Vranas, Evan Weinberg and Oliver Witzel
      \href{http://dx.doi.org/10.1103/PhysRevD.90.114502}{\textit{Physical Review D} \textbf{90}:114502} (2014) [\href{http://arxiv.org/abs/1405.4752}{arXiv:1405.4752}]
    \pagebreakitem
      \optem{$\mathcal N = 4$ supersymmetry on a space-time lattice} \\
      Simon Catterall, Poul H.~Damgaard, Thomas DeGrand, Joel Giedt and David Schaich \\
      \href{http://dx.doi.org/10.1103/PhysRevD.90.065013}{\textit{Physical Review D} \textbf{90}:065013} (2014) [\href{http://arxiv.org/abs/1405.0644}{arXiv:1405.0644}]
    \pagebreakitem
      \optem{Finite size scaling of conformal theories in the presence of a near-marginal operator} \\
      Anqi Cheng, Anna Hasenfratz, Yuzhi Liu, Gregory Petropoulos and David Schaich \\
      \href{http://dx.doi.org/10.1103/PhysRevD.90.014509}{\textit{Physical Review D} \textbf{90}:014509} (2014) [\href{http://arxiv.org/abs/1401.0195}{arXiv:1401.0195}]
    \pagebreakitem
      \optem{Maximum-likelihood approach to topological charge fluctuations in lattice gauge theory} \\
      LSD Collaboration: Richard C.~Brower, Michael Cheng, George T.~Fleming, Meifeng Lin, Ethan T.~Neil \textit{et al.} \\ %, James~C.~Osborn, Claudio Rebbi, Enrico Rinaldi, David~Schaich, Chris Schroeder, Gennady~Voronov, Pavlos Vranas, Evan Weinberg and Oliver Witzel
      \href{http://dx.doi.org/10.1103/PhysRevD.90.014503}{\textit{Physical Review D} \textbf{90}:014503} (2014) [\href{http://arxiv.org/abs/1403.2761}{arXiv:1403.2761}]%, featured in the July 2014 Physical Review D Kaleidoscope]
    \pagebreakitem
      \optem{Composite~bosonic~baryon~dark~matter~on~the~lattice:~$SU(4)$~baryon~spectrum~and~the~effective~Higgs~interaction} \\
      LSD Collaboration: Thomas Appelquist, Evan Berkowitz, Richard C.~Brower, Michael I.~Buchoff \textit{et al.} \\ %, George T.~Fleming, Joe Kiskis, Graham D.~Kribs, Meifeng Lin, Ethan T.~Neil, James C.~Osborn, Claudio Rebbi, Enrico Rinaldi, David Schaich, Chris Schroeder, Sergey Syritsyn, Gennady Voronov, Pavlos Vranas, Evan Weinberg, Oliver Witzel
      \href{http://dx.doi.org/10.1103/PhysRevD.89.094508}{\textit{Physical Review D} \textbf{89}:094508} (2014) [\href{http://arxiv.org/abs/1402.6656}{arXiv:1402.6656}]
    \pagebreakitem
      \optem{Improving the continuum limit of gradient flow step scaling} \\
      Anqi Cheng, Anna Hasenfratz, Yuzhi Liu, Gregory Petropoulos and David Schaich \\
      \href{http://dx.doi.org/10.1007/JHEP05(2014)137}{\textit{Journal of High Energy Physics} \textbf{1405}:137} (2014) [\href{http://arxiv.org/abs/1404.0984}{arXiv:1404.0984}]
    \pagebreakitem
      \optem{Two-Color Gauge Theory with Novel Infrared Behavior} \\
      LSD Collaboration: Thomas Appelquist, Richard C.~Brower, Michael I.~Buchoff, Michael Cheng \textit{et al.} \\ %, George T.~Fleming, Joe Kiskis, Meifeng Lin, Ethan T.~Neil, James~C.~Osborn, Claudio Rebbi, David~Schaich, Chris Schroeder, Sergey Syritsyn, Gennady~Voronov, Pavlos Vranas and Oliver Witzel
      \href{http://dx.doi.org/10.1103/PhysRevLett.112.111601}{\textit{Physical Review Letters} \textbf{112}:111601} (2014) [\href{http://arxiv.org/abs/1311.4889}{arXiv:1311.4889}]
    \pagebreakitem
      \optem{Scale-dependent mass anomalous dimension from Dirac eigenmodes} \\
      Anqi Cheng, Anna Hasenfratz, Gregory Petropoulos and David Schaich \\
      \href{http://dx.doi.org/10.1007/JHEP07(2013)061}{\textit{Journal of High Energy Physics} \textbf{1307}:061} (2013) [\href{http://arxiv.org/abs/1301.1355}{arXiv:1301.1355}]
    \pagebreakitem
      \optem{Lattice calculation of composite dark matter form factors} \\
      LSD Collaboration: Thomas Appelquist, Richard C.~Brower, Michael I.~Buchoff, Michael Cheng \textit{et al.} \\ %, Saul D.~Cohen, George T.~Fleming, Joe Kiskis, Meifeng Lin, Heechang Na, Ethan T.~Neil, James~C.~Osborn, Claudio Rebbi, David~Schaich, Chris Schroeder, Sergey Syritsyn, Gennady~Voronov, Pavlos Vranas and Joseph Wasem
      \href{http://dx.doi.org/10.1103/PhysRevD.88.014502}{\textit{Physical Review D} \textbf{88}:014502} (2013) [\href{http://arxiv.org/abs/1301.1693}{arXiv:1301.1693}]
    \pagebreakitem
      \optem{Novel phase in SU(3) lattice gauge theory with 12 light fermions} \\
      Anqi Cheng, Anna Hasenfratz and David Schaich \\
      \href{http://dx.doi.org/10.1103/PhysRevD.85.094509}{\textit{Physical Review D} \textbf{85}:094509} (2012) [\href{http://arxiv.org/abs/1111.2317}{arXiv:1111.2317}]
    \pagebreakitem
      \optem{WW scattering parameters via pseudoscalar phase shifts} \\
      LSD Collaboration: Thomas Appelquist, Ron Babich, Richard Brower, Michael I.~Buchoff \textit{et al.} \\ %, Michael Cheng, Michael A.~Clark, Saul D.~Cohen, George T.~Fleming, Joe Kiskis, Meifeng Lin, Ethan T.~Neil, James C.~Osborn, Claudio Rebbi, David Schaich, Sergey Syritsyn, Gennady~Voronov, Pavlos Vranas and Joseph Wasem
      \href{http://dx.doi.org/10.1103/PhysRevD.85.074505}{\textit{Physical Review D} \textbf{85}:074505} (2012) [\href{http://arxiv.org/abs/1201.3977}{arXiv:1201.3977}]
    \pagebreakitem
      \optem{Exploring strange nucleon form factors on the lattice} \\
      Ronald Babich, Richard Brower, Michael A.~Clark, George T.~Fleming, James C.~Osborn, Claudio Rebbi \textit{et al.} \\ % and David Schaich
      \href{http://dx.doi.org/10.1103/PhysRevD.85.054510}{\textit{Physical Review D} \textbf{85}:054510} (2012) [\href{http://arxiv.org/abs/1012.0562}{arXiv:1012.0562}]
    \pagebreakitem
      \optem{Lattice simulations and infrared conformality} \\
      Thomas Appelquist, George T.~Fleming, Meifeng Lin, Ethan T.~Neil and David Schaich \\
      \href{http://dx.doi.org/10.1103/PhysRevD.84.054501}{\textit{Physical Review D} \textbf{84}:054501} (2011) [\href{http://arxiv.org/abs/1106.2148}{arXiv:1106.2148}]
    \pagebreakitem
      \optem{Parity Doubling and the S Parameter below the Conformal Window} \\
      LSD Collaboration: Thomas Appelquist, Ron Babich, Richard Brower, Michael Cheng, Michael A.~Clark \textit{et al.} \\ %, Saul~D.~Cohen, George T.~Fleming, Joe Kiskis, Meifeng Lin, Ethan T.~Neil, James C.~Osborn, Claudio Rebbi, David Schaich and Pavlos Vranas
      \href{http://dx.doi.org/10.1103/PhysRevLett.106.231601}{\textit{Physical Review Letters} \textbf{106}:231601} (2011) [\href{http://arxiv.org/abs/1009.5967}{arXiv:1009.5967}]
    \pagebreakitem
      \optem{Toward TeV Conformality} \\
      LSD Collaboration: Thomas Appelquist, Adam Avakian, Ron Babich, Richard Brower, Michael Cheng \textit{et al.} \\ %, Michael~A.~Clark, Saul D.~Cohen, George T.~Fleming, Joseph Kiskis, Ethan~T.~Neil, James C.~Osborn, Claudio~Rebbi, David Schaich and Pavlos Vranas
      \href{http://dx.doi.org/10.1103/PhysRevLett.104.071601}{\textit{Physical Review Letters} \textbf{104}:071601} (2010) [\href{http://arxiv.org/abs/0910.2224}{arXiv:0910.2224}]
    \pagebreakitem
      \optem{Improved lattice measurement of the critical coupling in $\phi_2^4$ theory} \\
      David Schaich and Will Loinaz \\
      \href{http://dx.doi.org/10.1103/PhysRevD.79.056008}{\textit{Physical Review D} \textbf{79}:056008} (2009) [\href{http://arxiv.org/abs/0902.0045}{arXiv:0902.0045}] \\
% ------------------------------------------------------------------
%
%
%
% ------------------------------------------------------------------
% TODO: Page break hack
%\newpage \hspace{-22 pt}{\large \bfseries Other articles, theses \& white papers} \vspace{-8 pt}
\vspace{18 pt} \hspace{-22 pt}{\large \bfseries Other articles, theses \& white papers} \vspace{-8 pt}
    \pagebreakitem
      \optem{Lattice Gauge Theory for Physics Beyond the Standard Model} \\
      Richard C.~Brower, Anna Hasenfratz, Ethan Neil \textit{et al.} \\ %  Simon Catterall, George Fleming, Joel Giedt, Enrico Rinaldi, David Schaich, Evan Weinberg and Oliver Witzel
      \href{http://home.fnal.gov/~ask/USQCD/members/WP/BSMwhitepaper2018.pdf}{USQCD White Paper} (2019) [\href{http://arxiv.org/abs/1904.09964}{arXiv:1904.09964}] \\
      {[Published as \href{http://doi.org/10.1140/epja/i2019-12901-5}{\textit{European Physical Journal A} \textbf{55}:198} (2019)]}
    \pagebreakitem
      \optem{Lattice Gauge Theories at the Energy Frontier} \\
      Thomas Appelquist, Richard Brower, Simon Catterall, George Fleming, Joel Giedt, Anna Hasenfratz \textit{et al.} \\ % Julius Kuti, Ethan Neil and David Schaich
      \href{http://www.usqcd.org/documents/13BSM.pdf}{USQCD White Paper} (2013) [\href{http://arxiv.org/abs/1309.1206}{arXiv:1309.1206}]
    \pagebreakitem
      \optem{Approaching Conformality with Ten Flavors} \\
      LSD Collaboration: Thomas Appelquist, Richard C.~Brower, Michael I.~Buchoff, Michael Cheng \textit{et al.} \\ %, Saul D.~Cohen, George T.~Fleming, Joe Kiskis, Meifeng Lin, Heechang Na, Ethan T.~Neil, James~C.~Osborn, Claudio Rebbi, David~Schaich, Chris Schroeder, Gennady Voronov and Pavlos~Vranas
      \href{http://arxiv.org/abs/1204.6000}{arXiv:1204.6000} (2012)
    \pagebreakitem
      \optem{Strong Dynamics and Lattice Gauge Theory} \\
      David Schaich \\
      Ph.D.\ thesis, Boston University (2011) [\href{http://gradworks.umi.com/34/83/3483480.html}{UMI-3483480}]
    \pagebreakitem
      \optem{Hybrid Monte Carlo Simulation of Graphene on the Hexagonal Lattice} \\
      Richard C.~Brower, Claudio Rebbi and David Schaich \\
      \href{http://arxiv.org/abs/1101.5131}{arXiv:1101.5131} (2011)
    \pagebreakitem
      \optem{Lattice Simulations of Nonperturbative Quantum Field Theories} \\
      David Schaich \\
      B.A.\ thesis, Amherst College (2006) [\href{http://inspirehep.net/record/1386200}{INSPIRE-1386200}] \\
% ------------------------------------------------------------------
%
%
%
% ------------------------------------------------------------------
% TODO: Page break hack
%\newpage \hspace{-22 pt}{\large \bfseries Conference proceedings} \vspace{-8 pt}
\vspace{18 pt} \hspace{-22 pt}{\large \bfseries Conference proceedings} \vspace{-8 pt}
    \pagebreakitem
      \optem{Thermal phase structure of a supersymmetric matrix model} \\
      David Schaich, Raghav G.~Jha and Anosh Joseph \\
      \href{https://pos.sissa.it/363/069/}{\textit{Proceedings of Science} \textbf{LATTICE2019}:069} (2020) [\href{http://arxiv.org/abs/2003.01298}{arXiv:2003.01298}] % TODO: Change URL to DOI when latter available
    \pagebreakitem
      \optem{Stealth dark matter and gravitational waves} \\
      David Schaich \\
      \href{https://pos.sissa.it/363/068/}{\textit{Proceedings of Science} \textbf{LATTICE2019}:068} (2020) [\href{http://arxiv.org/abs/2002.00187}{arXiv:2002.00187}] % TODO: Change URL to DOI when latter available
    \pagebreakitem
      \optem{Exotic Phases of a Higgs--Yukawa Model with Reduced Staggered Fermions} \\
      Simon Catterall, Nouman Butt and David Schaich \\
      \href{https://pos.sissa.it/363/044/}{\textit{Proceedings of Science} \textbf{LATTICE2019}:044} (2020) [\href{http://arxiv.org/abs/2002.00034}{arXiv:2002.00034}] % TODO: Change URL to DOI when latter available
    \pagebreakitem
      \optem{Progress and prospects of lattice supersymmetry} \\
      David Schaich \\
      \href{https://doi.org/10.22323/1.334.0005}{\textit{Proceedings of Science} \textbf{LATTICE2018}:005} (2019) [\href{http://arxiv.org/abs/1810.09282}{arXiv:1810.09282}]
    \pagebreakitem
      \optem{Testing the holographic principle using lattice simulations} \\
      Raghav G.~Jha, Simon Catterall, David Schaich and Toby Wiseman \\
      \href{https://doi.org/10.1051/epjconf/201817508004}{\textit{European Physical Journal Web of Conferences} \textbf{175}:08004} (2018) [\href{http://arxiv.org/abs/1710.06398}{arXiv:1710.06398}]
    \pagebreakitem
      \optem{Phases of a strongly coupled four-fermion theory} \\
      David Schaich and Simon Catterall \\
      \href{https://doi.org/10.1051/epjconf/201817503004}{\textit{European Physical Journal Web of Conferences} \textbf{175}:03004} (2018) [\href{http://arxiv.org/abs/1710.08137}{arXiv:1710.08137}]
    \pagebreakitem
      \optem{Maximally supersymmetric Yang--Mills on the lattice} \\
      David Schaich and Simon Catterall \\
      \href{http://dx.doi.org/10.1142/9789813231467_0028}{\textit{Origin of Mass and Strong Coupling Gauge Theories}:199} (2018) [\href{http://arxiv.org/abs/1508.00884}{arXiv:1508.00884}] \\
      {[Reprinted as \href{http://dx.doi.org/10.1142/S0217751X17470194}{\textit{International Journal of Modern Physics A} \textbf{32}:1747019} (2017)]}
    \pagebreakitem
      \optem{Finite-temperature study of eight-flavor SU(3) gauge theory} \\
      David Schaich, Anna Hasenfratz and Enrico Rinaldi for the Lattice Strong Dynamics (LSD) Collaboration \\
      \href{http://dx.doi.org/10.1142/9789813231467_0051}{\textit{Origin of Mass and Strong Coupling Gauge Theories}:351} (2018) [\href{http://arxiv.org/abs/1506.08791}{arXiv:1506.08791}]
    \pagebreakitem
      \optem{Latest results from lattice $\mathcal N = 4$ supersymmetric Yang--Mills} \\
      David Schaich, Simon Catterall, Poul H.~Damgaard and Joel Giedt \\
      \href{https://doi.org/10.22323/1.256.0221}{\textit{Proceedings of Science} \textbf{LATTICE2016}:221} (2016) [\href{http://arxiv.org/abs/1611.06561}{arXiv:1611.06561}]
    \pagebreakitem
      \optem{S-duality in lattice super Yang--Mills} \\
      Joel Giedt, Simon Catterall, Poul Damgaard and David Schaich \\
      \href{https://doi.org/10.22323/1.256.0209}{\textit{Proceedings of Science} \textbf{LATTICE2016}:209} (2016) [\href{http://arxiv.org/abs/1804.07792}{arXiv:1804.07792}]
    \pagebreakitem
      \optem{Aspects of lattice $\mathcal N = 4$ supersymmetric Yang--Mills} \\
      David Schaich \\
      \href{https://doi.org/10.22323/1.251.0242}{\textit{Proceedings of Science} \textbf{LATTICE 2015}:242} (2015) [\href{http://arxiv.org/abs/1512.01137}{arXiv:1512.01137}]
    \pagebreakitem
      \optem{Results from lattice simulations of $\mathcal N = 4$ supersymmetric Yang--Mills} \\
      Simon Catterall, Joel Giedt, David Schaich, Poul H.~Damgaard and Thomas DeGrand \\
      \href{https://doi.org/10.22323/1.214.0267}{\textit{Proceedings of Science} \textbf{LATTICE2014}:267} (2014) [\href{http://arxiv.org/abs/1411.0166}{arXiv:1411.0166}]
    \pagebreakitem
      \optem{Reaching the chiral limit in many flavor systems} \\
      Anna Hasenfratz, Anqi Cheng, Gregory Petropoulos and David Schaich \\
      \href{http://dx.doi.org/10.1142/9789814566254_0004}{\textit{Strong Coupling Gauge Theories in the LHC Perspective}:44} (2014) [\href{http://arxiv.org/abs/1303.7129}{arXiv:1303.7129}]
    \pagebreakitem
      \optem{Improved Lattice Renormalization Group Techniques} \\
      Gregory Petropoulos, Anqi Cheng, Anna Hasenfratz and David Schaich \\
      \href{https://doi.org/10.22323/1.187.0079}{\textit{Proceedings of Science} \textbf{LATTICE 2013}:079} (2013) [\href{http://arxiv.org/abs/1311.2679}{arXiv:1311.2679}]
    \pagebreakitem
      \optem{Determining the mass anomalous dimension through the eigenmodes of Dirac operator} \\
      Anqi Cheng, Anna Hasenfratz, Gregory Petropoulos and David Schaich \\
      \href{https://doi.org/10.22323/1.187.0088}{\textit{Proceedings of Science} \textbf{LATTICE 2013}:088} (2013) [\href{http://arxiv.org/abs/1311.1287}{arXiv:1311.1287}]
    \pagebreakitem
      \optem{Eight light flavors on large lattice volumes} \\
      David Schaich for \href{http://bsm.physics.yale.edu}{USBSM} \\
      \href{https://doi.org/10.22323/1.187.0072}{\textit{Proceedings of Science} \textbf{LATTICE 2013}:072} (2013) [\href{http://arxiv.org/abs/1310.7006}{arXiv:1310.7006}]
    \pagebreakitem
      \optem{Finite size scaling and the effect of the gauge coupling in 12 flavor systems} \\
      Anna Hasenfratz, Anqi Cheng, Gregory Petropoulos and David Schaich \\
      \href{https://doi.org/10.22323/1.187.0075}{\textit{Proceedings of Science} \textbf{LATTICE 2013}:075} (2013) [\href{http://arxiv.org/abs/1310.1124}{arXiv:1310.1124}]
    \pagebreakitem
      \optem{MCRG study of 8 and 12 fundamental flavors} \\
      Gregory Petropoulos, Anqi Cheng, Anna Hasenfratz and David Schaich \\
      \href{https://doi.org/10.22323/1.164.0051}{\textit{Proceedings of Science} \textbf{Lattice 2012}:051} (2012) [\href{http://arxiv.org/abs/1212.0053}{arXiv:1212.0053}]
    \pagebreakitem
      \optem{Bulk and finite-temperature transitions in SU(3) gauge theories with many light fermions} \\
      David Schaich, Anqi Cheng, Anna Hasenfratz and Gregory Petropoulos \\
      \href{https://doi.org/10.22323/1.164.0028}{\textit{Proceedings of Science} \textbf{Lattice 2012}:028} (2012) [\href{http://arxiv.org/abs/1207.7164}{arXiv:1207.7164}]
    \pagebreakitem
      \optem{Mass anomalous dimension from Dirac eigenmode scaling in conformal and confining systems} \\
      Anna Hasenfratz, Anqi Cheng, Gregory Petropoulos and David Schaich \\
      \href{https://doi.org/10.22323/1.164.0034}{\textit{Proceedings of Science} \textbf{Lattice 2012}:034} (2012) [\href{http://arxiv.org/abs/1207.7162}{arXiv:1207.7162}]
    \pagebreakitem
      \optem{Strange nucleon form factors on 2+1f anisotropic wilson clover lattices} \\
      Michael Cheng, Ronald Babich, Richard Brower, Michael A.~Clark, Saul D.~Cohen, George T.~Fleming \textit{et al.} \\
      \href{https://doi.org/10.22323/1.164.0166}{\textit{Proceedings of Science} \textbf{Lattice 2012}:166} (2012)
    \pagebreakitem
      \optem{Hybrid Monte Carlo simulation on the graphene hexagonal lattice} \\
      Richard Brower, Claudio Rebbi and David Schaich \\
      \href{https://doi.org/10.22323/1.139.0056}{\textit{Proceedings of Science} \textbf{Lattice 2011}:056} (2011) [\href{http://arxiv.org/abs/1204.5424}{arXiv:1204.5424}]
    \pagebreakitem
      \optem{S parameter and parity doubling below the conformal window} \\
      David Schaich for the Lattice Strong Dynamics (LSD) Collaboration \\
      \href{https://doi.org/10.22323/1.139.0087}{\textit{Proceedings of Science} \textbf{Lattice 2011}:087} (2011) [\href{http://arxiv.org/abs/1111.4993}{arXiv:1111.4993}]
    \pagebreakitem
      \optem{Lattice study of ChPT beyond QCD} \\
      LSD Collaboration: Ethan T.~Neil, Adam Avakian, Ron Babich, Richard C.~Brower, Michael Cheng \textit{et al.} \\ %, Michael~A.~Clark, Saul D.~Cohen, George T.~Fleming, Joseph Kiskis, James C.~Osborn, Claudio Rebbi, David~Schaich and Pavlos Vranas
      \href{https://doi.org/10.22323/1.086.0088}{\textit{Proceedings of Science} \textbf{CD09}:088} (2009) [\href{http://arxiv.org/abs/1002.3777}{arXiv:1002.3777}]
    \pagebreakitem
      \optem{M\"obius Algorithm for Domain Wall and GapDW Fermions} \\
      Richard Brower, Ron Babich, Kostas Orginos, Claudio Rebbi, David Schaich and Pavlos Vranas \\
      \href{https://doi.org/10.22323/1.066.0034}{\textit{Proceedings of Science} \textbf{LATTICE 2008}:034} (2008) [\href{http://arxiv.org/abs/0906.2813}{arXiv:0906.2813}]
  \end{revnumerate}
% ------------------------------------------------------------------
%
%
%
% ------------------------------------------------------------------
% TODO: Page break hack
%\newpage {\large \bfseries In active preparation} \hfill(ordered by anticipated completion)
\vspace{20 pt} {\large \bfseries In active preparation} \hfill(ordered by anticipated completion)
  \pagebreakitem
    \optem{Three dimensional $\mathcal N = 8$ super-Yang--Mills theory on the lattice and dual black branes} \\
    Simon Catterall, Joel Giedt, Raghav G.~Jha, David Schaich and Toby Wiseman \\
  \pagebreakitem
    \optem{Nonperturbative ingredients for stochastic gravitational wave predictions} \\
    Venkitesh Ayyar, Ethan T.~Neil, Enrico Rinaldi and David Schaich \\
  \pagebreakitem
    \optem{Plane wave matrix model at finite temperatures on the lattice} \\
    Raghav G.~Jha, Anosh Joseph and David Schaich \\
\begin{comment} % Don't have to prove activity at the moment
  \pagebreakitem
    \optem{Improved parallel software for lattice supersymmetry} \\
    David Schaich, Georg Bergner, Simon Catterall, Raghav G.~Jha and Anosh Joseph \\
  \pagebreakitem
    \optem{Tests of the twelve-flavor beta function with staggered fermions} \\
    Anna Hasenfratz and David Schaich \\
  \pagebreakitem
    \optem{Lattice calculations of the $\mathcal N = 4$ supersymmetric Yang--Mills static potential} \\
    Simon Catterall, Poul H.~Damgaard, Joel Giedt, Raghav G.~Jha and David Schaich
  \pagebreakitem
    \optem{Mass anomalous dimension of many-flavor systems with staggered fermions} \\
    Anna Hasenfratz and David Schaich \\[24 pt]
  \pagebreakitem
    \optem{$\mathcal N = 4$ supersymmetric Yang--Mills anomalous dimensions from nonperturbative lattice calculations} \\
    Simon Catterall, Poul H.~Damgaard, Joel Giedt and David Schaich \\
%  $\S$: Conference proceedings \\[2 pt]
%  $\dag$: Journal article
\end{comment}
\end{spacelist}
% ------------------------------------------------------------------

% ------------------------------------------------------------------



% ------------------------------------------------------------------
\newpage
\makeheading{Grants and computing allocations}
% ------------------------------------------------------------------
% For \input only
% Suppress indentation
\renewenvironment{spacelist} {
  \begin{list} {} {
    \setlength{\topsep}{-8 pt}
    \setlength{\itemsep}{5 pt}
    \setlength{\leftmargin}{-1.15 in}
  }
}{
  \end{list}
}

\raggedright
\vspace{12 pt}
\begin{spacelist}
  \item {\large \bfseries Monetary awards as principal investigator or a primary author of proposal} \\[10 pt]
    \begin{tabular}[t]{cl}
      2019 & \href{https://www.ukri.org}{UK Research and Innovation} Future Leader Fellowship {MR/S015418/1}, 943k GBP \\
           & \textit{New Frontiers of Lattice Field Theory}                                                            \\[6 pt]
      2018 & University of Bern Faculty of Science conference travel grant, 1700 CHF                                   \\[6 pt]
      2011 & National Science Foundation Award OISE-1107903, 5700 USD                                                  \\
           & \textit{Exploring the Origin of Mass with High-Performance Computing}                                     \\
    \end{tabular}

  \spacer
  \item {\large \bfseries Monetary awards as participating investigator} \\[10 pt]
    \begin{tabular}[t]{cl}
      2020 & \href{https://stfc.ukri.org}{Science and Technology Facilities Council} Consolidated Grant {ST/T000988/1}, 898k GBP \\
           & \textit{New Horizons in Quantum Field Theory, Particle Physics and String Phenomenology}                            \\[6 pt]
      2020 & Science and Technology Facilities Council Virtual Centre award {ST/T000813/1}, 19.5k GBP                            \\
           & \textit{UK Lattice Field Theory Virtual Centre}                                                                     \\[6 pt]
      2020 & University of Liverpool Research Centre in Mathematics and Modelling workshop grant, 3000 GBP                       \\
           & \textit{Physics of strongly interacting fermions in low-dimensional materials}                                      \\
           & \hspace{2.4 cm} \textit{and high-energy physics: a numerical perspective} (cancelled due to COVID-19 pandemic)      \\
    \end{tabular}

  \vspace{18 pt}
  \item {\large \bfseries Monetary awards as project partner} \\[10 pt]
    \begin{tabular}[t]{cl}
      2020 & \href{https://epsrc.ukri.org}{Engineering and Physical Sciences Research Council} grant {EP/V001329/1}, 94k GBP \\
           &  Exascale Computing Algorithms and Infrastructures Benefitting UK Research (ExCALIBUR) programme                \\
           & \textit{Lattice Field Theory at the Exascale Frontier}                                                          \\
    \end{tabular}

  \vspace{18 pt}
  \item {\large \bfseries Computing allocations as principal investigator or a primary author of proposal} \\[10 pt]
    \begin{tabular}[t]{cl}
      2022 & \href{https://dirac.ac.uk}{DiRAC} proposal under review requesting 25.3M core-hours                      \\ % Cambridge
           & \textit{Lattice studies of 3d super-Yang--Mills and holography}                                          \\[6 pt]
      2021 & DiRAC computing allocation, 20.78M core hours + 2.74M core-hour uplift                                   \\ % Cambridge (uplift Durham)
           & \textit{Electroweak phenomenology from lattice strong dynamics}                                          \\[6 pt]
      2019 & \href{http://www.usqcd.org}{USQCD} computing allocation, 2.25M core hours                                \\ % Fermilab
           & \textit{Phases of a Higgs--Yukawa Theory and Symmetric Mass Generation}                                  \\[6 pt]
      2017 & USQCD computing allocation, 2.52M core hours                                                             \\ % Fermilab
           & \textit{Exploring Improved Methods to Extract the $0^{++}$ Mass in an SU(3) Gauge Theory with 8 Flavors} \\[6 pt]
      2017 & USQCD computing allocation, 12.5M core hours                                                             \\ % Fermilab
           & \textit{Thermodynamics of 3D Supersymmetric Yang--Mills}                                                 \\[6 pt]
      2017 & USQCD computing allocation, 2.46M core hours                                                             \\ % JLab
           & \textit{A new critical point in lattice four-fermion theories?}                                          \\[6 pt]
      2016 & \href{http://www.xsede.org}{XSEDE} computing allocation, 4.11M core hours                                \\ % Comet at SDSC
           & \textit{Lattice studies of supersymmetric gauge theories}                                                \\[6 pt]
      2016 & USQCD computing allocation, 11.37M core hours                                                            \\ % Fermilab
           & \textit{Lattice $\mathcal N = 4$ supersymmetric Yang--Mills on the Coulomb branch}                       \\[6 pt]
      2015 & USQCD computing allocation, 10.71M core hours                                                            \\ % Fermilab
           & \textit{Anomalous dimensions from lattice $\mathcal N = 4$ super Yang--Mills with an improved action}    \\[6 pt]
      2014 & USQCD computing allocation, 11.04M core hours                                                            \\ % Fermilab
           & \textit{Lattice $\mathcal N = 4$ supersymmetric Yang--Mills with 2, 3 and 4 colors}                      \\[6 pt]
      2013 & \href{https://www.top500.org/system/176922}{Janus} computing allocation, 4.8M core hours                 \\ % University of Colorado
           & \textit{Lattice studies of an infrared-conformal gauge theory}                                           \\[6 pt]
    \end{tabular} % TODO: Page break hack
    \begin{tabular}[t]{cl}
      2013 & USQCD computing allocation, 9.97M core hours                                                             \\ % Fermilab
           & \textit{Eight-flavor SU(3) gauge theory with nHYP-smeared fermions}                                      \\[6 pt]
      2013 & XSEDE computing allocation, 7.0M core hours                                                              \\ % 3.5 on Gordon at SDSC, 3.5 on Stampede at TACC
           & \textit{SU(3) gauge theories with many fermions --- to the chiral limit}                                 \\[6 pt]
      2012 & Janus computing allocation, 5.5M core hours                                                              \\ % University of Colorado
           & \textit{Lattice studies of strongly-interacting gauge theories with many light fermions}                 \\[6 pt]
      2012 & USQCD computing allocation, 4.84M core hours                                                             \\ % Fermilab
           & \textit{Many flavor gauge theories: finite volume scaling at small masses}                               \\[6 pt]
      2011 & XSEDE computing allocation, 2.5M core hours                                                              \\ % Gordon at SDSC
           & \textit{Phase structure of SU(3) gauge theory with many light fermions}                                  \\
    \end{tabular}

  \vspace{18 pt}
  \item {\large \bfseries Computing allocations as participating investigator} \\[10 pt]
    \begin{tabular}[t]{cl}
      2021 & USQCD computing allocation, 15M core hours                                                           \\ % JLab (KNL)
           & \textit{Composite Higgs model with four light and six heavy flavors}                                 \\[6 pt]
      2020 & USQCD computing allocation, 13.2M core hours                                                         \\ % JLab (KNL)
           & \textit{Composite Higgs model with four light and six heavy flavors}                                 \\[6 pt]
      2020 & USQCD computing allocation, 1.07M core hours                                                         \\ % Fermilab
           & \textit{Decoupling doublers using generalized Yukawa interactions}                                   \\[6 pt]
      2019 & USQCD computing allocation, 8.75M core hours                                                         \\ % BNL (KNL)
           & \textit{Composite Higgs model with four light and six heavy flavors}                                 \\[6 pt]
      2018 & USQCD computing allocation, 12M core hours                                                           \\ % Fermilab
           & \textit{Thermodynamics of SYM theory in three and four dimensions}                                   \\[6 pt]
      2018 & USQCD computing allocation, 29M core hours                                                           \\ % BNL (KNL)
           & \textit{Composite Higgs model with four light and six heavy flavors}                                 \\[6 pt]
      2017 & USQCD computing allocation, 12M core hours                                                           \\ % JLab (KNL)
           & \textit{Simulations of four light and six heavy flavors using smeared M\"obius domain-wall fermions} \\[6 pt]
      2016 & USQCD computing allocation, 9.51M core hours                                                         \\ % Fermilab
           & \textit{Measuring the Low Energy Effective Theory in Multiflavor QCD}                                \\[6 pt]
      2016 & ASCR Leadership Computing Challenge allocation, 55M core hours                                       \\ % Mira at Argonne
           & \textit{Exploring Higgs Compositeness Mechanism in the Era of the 14 TeV LHC}                        \\[6 pt]
      2015 & USQCD computing allocation, 8.6M core hours                                                          \\ % Fermilab
           & \textit{Non-Perturbative Collider Phenomenology of Stealth Dark Matter}                              \\[6 pt]
      2014 & USQCD computing allocation, 13.33M core hours                                                        \\ % Fermilab
           & \textit{Electromagnetic Polarizability of Bosonic Composite Dark Matter}                             \\[6 pt]
      2013 & USQCD computing allocation, 29k GPU hours                                                            \\ % Fermilab
           & \textit{$\mathcal N = 4$ Super Yang--Mills on GPUs}                                                  \\[6 pt]
      2013 & USQCD computing allocation, 9.71M core hours                                                         \\ % Fermilab
           & \textit{Lattice study of $\mathcal N = 4$ Super Yang--Mills}                                         \\[6 pt]
      2013 & USQCD computing allocation, 9.35M core hours                                                         \\ % Fermilab
           & \textit{Two-Color Gauge Theories in the Higgs Era}                                                   \\[6 pt]
      2013 & Janus computing allocation, 1.6M core hours                                                          \\ % University of Colorado
           & \textit{Finite size scaling studies with twelve light fermions}                                      \\[6 pt]
      2013 & XSEDE computing allocation, 9.2M core hours                                                          \\ % Kraken
           & \textit{Many-Fermion Gauge Theories for TeV Physics}                                                 \\[6 pt]
      2012 & USQCD computing allocation, 12.4M core hours                                                         \\ % Fermilab
           & \textit{Extended study of many fermion gauge theories for TeV physics}                               \\[6 pt]
      2012 & USQCD computing allocation, 86.3k GPU hours                                                          \\ % JLab
           & \textit{Disconnected contributions to nucleon form factors with chiral fermions}                     \\[6 pt]
    \end{tabular} % TODO: Page break hack
    \begin{tabular}[t]{cl}
      2011 & USQCD computing allocation, 12.08M core hours                                                        \\ % JLab
           & \textit{Exploration of Many-Fermion Gauge Theories for TeV Physics}                                  \\[6 pt]
      2010 & USQCD computing allocation, 5M core hours                                                            \\ % JLab(?)
           & \textit{Two-Color Gauge Theories for TeV Physics}                                                    \\[6 pt]
      2009 & USQCD computing allocation, 580k core hours                                                          \\ % Fermilab
           & \textit{Strange quark contribution to nucleon form factors}                                          \\[6 pt]
      2008 & USQCD computing allocation, 488k core hours                                                          \\ % Fermilab
           & \textit{QCD Vacuum Polarization Contribution to the S Parameter and g-2}                             \\
    \end{tabular}
\end{spacelist}
% ------------------------------------------------------------------

% ------------------------------------------------------------------



% ------------------------------------------------------------------
\newpage
\makeheading{Presentations}
% ------------------------------------------------------------------
% For \input only
% Suppress indentation
\renewenvironment{spacelistout} {
  \begin{list} {} {
    \setlength{\topsep}{-8 pt}
    \setlength{\itemsep}{5 pt}
    \setlength{\leftmargin}{-1.15 in}
  }
}{
  \end{list}
}

\raggedright
\vspace{12 pt}
\begin{spacelistout}
  \item {\large \bfseries Invited talks}
  \begin{revnumerate}
  \setlength{\topsep}{-8 pt}
  \setlength{\itemsep}{10 pt}
  \setlength{\leftmargin}{0 mm}
    \pagebreakitem
      \textit{\href{http://www.davidschaich.net/talks/1811Liverpool.pdf}{Lattice studies of maximally supersymmetric Yang--Mills theories}} \\
      University of Liverpool Theoretical Physics Seminar, 28 November 2018
    \pagebreakitem
      \textit{\href{http://www.davidschaich.net/talks/1811Swansea.pdf}{Maximally supersymmetric Yang--Mills on the lattice}} \\
      Swansea University Theory Seminar, 23 November 2018
    \pagebreakitem
      \textit{\href{http://www.davidschaich.net/talks/1811Stavanger.pdf}{Composite dark matter and the role of lattice field theory}} \\
      University of Stavanger Physics Seminar, 1 November 2018
    \pagebreakitem
      \textit{\href{http://www.davidschaich.net/talks/1809ECT.pdf}{Lattice $\mathcal N = 4$ Supersymmetric Yang--Mills}} \\
      Workshop on Quantum Gravity meets Lattice QFT, ECT* Trento, 5 September 2018
    \pagebreakitem
      \textit{\href{http://www.davidschaich.net/talks/1807Lattice.pdf}{Progress and prospects of lattice supersymmetry}} \\
      36th International Symposium on Lattice Field Theory, Michigan State University, 24 July 2018
    \pagebreakitem
      \textit{\href{http://www.davidschaich.net/talks/1807Liverpool.pdf}{Physics Out Of The Box: The Impact of Lattice Field Theory}} \\
      University of Liverpool, 5 July 2018
    \pagebreakitem
      \textit{\href{http://www.davidschaich.net/talks/1806CERN.pdf}{Lattice studies of maximally supersymmetric Yang--Mills theories}} \\
      CERN Lattice Seminar, 7 June 2018
    \pagebreakitem
      \textit{\href{http://www.davidschaich.net/talks/1803FIU.pdf}{Physics Out Of The Box: Frontiers of Lattice Field Theory}} \\
      Florida International University Colloquium, 9 March 2018
    \pagebreakitem
      \textit{\href{http://www.davidschaich.net/talks/1801Bangalore.pdf}{Lattice $\mathcal N = 4$ Supersymmetric Yang--Mills}} \\
      Program on Nonperturbative and Numerical Approaches to Quantum Gravity, String Theory and Holography, \\ International Centre for Theoretical Sciences, Bangalore, 31 January 2018
    \pagebreakitem
      \textit{\href{http://www.davidschaich.net/talks/1801StonyBrook.pdf}{Maximally supersymmetric Yang--Mills on the lattice}} \\
      Workshop on Continuum and Lattice Approaches to the IR Behavior of (Quasi-)Conformal Gauge Theories, \\ Simons Center for Geometry and Physics, Stony Brook, 11 January 2018
    \pagebreakitem
      \textit{\href{http://www.davidschaich.net/talks/1712Montpellier.pdf}{Lattice gauge theory at the electroweak scale}} \\
      Workshop on Strong Dynamics at the Electroweak Scale, University of Montpellier, 6 December 2017
    \pagebreakitem
      \textit{\href{http://www.davidschaich.net/talks/1711Jena.pdf}{Lattice studies of maximally supersymmetric Yang--Mills theories}} \\
      Workshop on Strongly Interacting Field Theories, Friedrich Schiller University Jena, 25 November 2017
    \pagebreakitem
      \textit{\href{http://www.davidschaich.net/talks/1705Planck.pdf}{Lattice gauge theory beyond the standard model}} \\
      20th International Conference From the Planck Scale to the Electroweak Scale, Warsaw, 22 May 2017
    \pagebreakitem
      \textit{\href{http://www.davidschaich.net/talks/1611Edinburgh.pdf}{Maximally supersymmetric Yang--Mills on the lattice}} \\
      University of Edinburgh Higgs Centre Particle Physics Theory Seminar, 23 November 2016
    \pagebreakitem
      \textit{\href{http://www.davidschaich.net/talks/1604Glasgow.pdf}{Physics Out Of The Box: The impact of lattice gauge theory}} \\
      University of Glasgow, 18 April 2016
    \pagebreakitem
      \textit{\href{http://www.davidschaich.net/talks/1602RPI.pdf}{Composite dark matter and the role of lattice field theory}} \\
      Rensselaer Polytechnic Institute Colloquium, 17 February 2016
    \pagebreakitem
      \textit{\href{http://www.davidschaich.net/talks/1512Jena.pdf}{Maximally supersymmetric Yang--Mills on the lattice}} \\
      Friedrich Schiller University Jena Quantum Theory Seminar, 17 December 2015
    \pagebreakitem
      \textit{\href{http://www.davidschaich.net/talks/151123Humboldt.pdf}{Electroweak Phenomenology and Lattice Strong Dynamics}} \\
      Humboldt / DESY Lattice Seminar, 23 November 2015
    \pagebreakitem
      \textit{\href{http://www.davidschaich.net/talks/151118Humboldt.pdf}{$\mathcal N = 4$ supersymmetric Yang--Mills on a space-time lattice}} \\
      Humboldt University QFT / String Seminar, 18 November 2015
    \pagebreakitem
      \textit{\href{http://www.davidschaich.net/talks/1511StonyBrook.pdf}{Physics Out Of The Box: The impact of lattice gauge theory and advanced computing}} \\
      Stony Brook University Nuclear Theory Seminar, 13 November 2015
    \pagebreakitem
      \textit{\href{http://www.davidschaich.net/talks/1511MSU.pdf}{Physics Out Of The Box: The impact of lattice gauge theory and large-scale computing}} \\
      Michigan State University High Energy Physics Seminar, 3 November 2015
    \pagebreakitem
      \textit{\href{http://www.davidschaich.net/talks/1509KITP.pdf}{Lattice Gauge Theory for N=4 Super Yang--Mills}} \\
      Lattice Gauge Theory for the LHC and Beyond, \\ Kavli Institute for Theoretical Physics, Santa Barbara, 16 September 2015
    \pagebreakitem
      \textit{\href{http://www.davidschaich.net/talks/LGT4CH.pdf}{Lattice gauge theory for composite Higgs}} \\
      23rd International Conference on Supersymmetry and Unification of Fundamental Interactions, \\ Lake Tahoe, CA, 28 August 2015
    \pagebreakitem
      \textit{\href{http://www.davidschaich.net/talks/Aspen15.pdf}{Lattice supersymmetry in a nutshell}} \\
      Understanding Strongly Coupled Systems in High Energy and Condensed Matter Physics, \\ Aspen Center for Physics, 28 May 2015
    \pagebreakitem
      \textit{\href{http://www.davidschaich.net/talks/Livermore1504.pdf}{Lattice for Supersymmetric Physics}} \\
      Lattice for Beyond the Standard Model Physics, Lawrence Livermore National Laboratory, 24 April 2015
    \pagebreakitem
      \textit{\href{http://www.davidschaich.net/talks/Purdue1504.pdf}{Strong Dynamics and Lattice Gauge Theory: Going Beyond QCD}} \\
      Purdue High Energy Theory Seminar, 7 April 2015
    \pagebreakitem
      \textit{\href{http://www.davidschaich.net/talks/SCGT15.pdf}{Maximally supersymmetric Yang--Mills on the lattice}} \\
      Origin of Mass and Strong Coupling Gauge Theories, \\ Kobayashi--Maskawa Institute, Nagoya University, 5 March 2015
    \pagebreakitem
      \textit{\href{http://www.davidschaich.net/talks/Yale1502.pdf}{$\mathcal N = 4$ supersymmetric Yang--Mills on a space-time lattice}} \\
      Yale Particle Theory Seminar, 10 February 2015
    \pagebreakitem
      \textit{\href{http://www.davidschaich.net/talks/USQCD14.pdf}{Status and prospects for supersymmetry on the lattice}} \\
      USQCD All Hands Meeting, Jefferson Lab, 19 April 2014
    \pagebreakitem
      \textit{\href{http://www.davidschaich.net/talks/Mass13.pdf}{Fun with the $S$ parameter on the lattice}} \\
      Origin of Mass 2013 Lattice BSM Workshop, CP$^3$-Origins, Odense, Denmark, 7 August 2013
    \pagebreakitem
      \textit{\href{http://www.davidschaich.net/talks/Aspen13.pdf}{Exploring a new lattice phase}} \\
      Lattice Gauge Theory in the LHC Era, Aspen Center for Physics, 31 May 2013
    \pagebreakitem
      \textit{\href{http://www.davidschaich.net/talks/SU1304.pdf}{Going Beyond QCD on the Lattice}} \\
      Syracuse University High Energy Theory Seminar, 23 April 2013
    \pagebreakitem
      \textit{\href{http://www.davidschaich.net/talks/LME2012.pdf}{SU(3) gauge theories with many massless fermions: methods and mysteries}} \\
      Lattice~Meets~Experiment:~Beyond~the~Standard~Model,~University~of~Colorado,~27~October~2012
    \pagebreakitem
      \textit{\href{http://www.davidschaich.net/talks/LME2011.pdf}{The $S$ Parameter on the Lattice}} \\
      Lattice Meets Experiment: Beyond the Standard Model, Fermilab, 15 October 2011
    \pagebreakitem
      \textit{\href{http://www.davidschaich.net/talks/1104CCS.pdf}{Lattice QCD -- and Beyond}} \\
      Boston University Center for Computational Science Seminar, 29 April 2011
    \pagebreakitem
  %    {\small \textit{\href{http://www.davidschaich.net/talks/EWSB.pdf}{Electroweak Symmetry Breaking: An enduring mystery of the standard model of particle physics, and how we hope to solve it}}} \\
      \textit{\href{http://www.davidschaich.net/talks/EWSB.pdf}{Electroweak Symmetry Breaking}} \\
      Amherst College Colloquium, 1 October 2009 \\
% ------------------------------------------------------------------
%
%
%
% ------------------------------------------------------------------
%\newpage
\vspace{18 pt}
\hspace{-22 pt}{\large \bfseries Contributed talks} \vspace{-8 pt}
    \pagebreakitem
      \textit{\href{http://www.davidschaich.net/talks/1803Bern.pdf}{Lower-dimensional lattice supersymmetry}} \\
      University of Bern AEC Institute for Theoretical Physics lunch seminar, 22 March 2018
    \pagebreakitem
      \textit{\href{http://www.davidschaich.net/talks/lattice17.pdf}{Phases of a strongly coupled four-fermion theory}} \\
      35th International Symposium on Lattice Field Theory, Granada, Spain, 22 June 2017
    \pagebreakitem
      \textit{\href{http://www.davidschaich.net/talks/1702BadHonnef.pdf}{Light scalar from lattice strong dynamics}} \\
      637th Wilhelm und Else Heraeus-Seminar ``Understanding the LHC'', Bad Honnef, Germany, 14 February 2017
    \pagebreakitem
      \textit{\href{http://www.davidschaich.net/talks/latticeN4_Bern.pdf}{Lattice $\mathcal N = 4$ SYM}} \\
      University of Bern AEC Institute for Theoretical Physics lunch seminar, 13 October 2016
    \pagebreakitem
      \textit{\href{http://www.davidschaich.net/talks/lattice16.pdf}{Latest results from lattice $\mathcal N = 4$ super Yang--Mills}} \\
      34th International Symposium on Lattice Field Theory, Southampton, England, 26 July 2016
    \pagebreakitem
      \textit{\href{http://www.davidschaich.net/talks/SUSY15.pdf}{$\mathcal N = 4$ super Yang--Mills on a space-time lattice}} \\
      23rd International Conference on Supersymmetry and Unification of Fundamental Interactions, \\ Lake Tahoe, CA, 25 August 2015
    \pagebreakitem
      \textit{\href{http://www.davidschaich.net/talks/lattice15.pdf}{New results from lattice $\mathcal N = 4$ super Yang--Mills}} \\
      33rd International Symposium on Lattice Field Theory, Kobe, Japan, 18 July 2015
    \pagebreakitem
      \textit{\href{http://www.davidschaich.net/talks/lattice14.pdf}{Results from lattice studies of maximally supersymmetric Yang--Mills}} \\
      32nd International Symposium on Lattice Field Theory, Columbia University, 25 June 2014
    \pagebreakitem
      \textit{\href{http://www.davidschaich.net/talks/LSD_SU4_1403.pdf}{Composite dark matter on the lattice: the effective Higgs interaction}} \\
      Syracuse University High Energy Theory Seminar, 24 March 2014
    \pagebreakitem
      \textit{\href{http://www.davidschaich.net/talks/LSD_EFT13.pdf}{From Lattice Strong Dynamics to Electroweak Phenomenology}} \\
      Syracuse University High Energy Theory Seminar, 4 November 2013
    \pagebreakitem
      \textit{\href{http://www.davidschaich.net/talks/lattice13.pdf}{Eight light flavors on large lattice volumes}} \\
      31st International Symposium on Lattice Field Theory, Mainz, Germany, 29 July 2013
    \pagebreakitem
      \textit{\href{http://www.davidschaich.net/talks/April13.pdf}{Lattice calculation of composite dark matter form factors}} \\
      APS April Meeting, Denver, 13 April 2013
    \pagebreakitem
      \textit{\href{http://www.davidschaich.net/talks/g-2_1209.pdf}{$(g - 2)_{\mu}$ FAQ}} \\
      University of Colorado, 20 September 2012
    \pagebreakitem
      \textit{\href{http://www.davidschaich.net/talks/lattice12.pdf}{Bulk and finite-temperature transitions in SU(3) gauge theories with many light fermions}} \\
      30th International Symposium on Lattice Field Theory, Cairns, Australia, 25 June 2012
    \pagebreakitem
      \textit{\href{http://www.davidschaich.net/talks/LSD10f_1204.pdf}{Lattice Strong Dynamics: Turning it up to ten}} \\
      University of Colorado, 19 April 2012
    \pagebreakitem
      \textit{\href{http://www.davidschaich.net/talks/April12}{Novel phase in SU(3) lattice gauge theories with many light fermions}} \\
      APS April Meeting, Atlanta, 1 April 2012
    \pagebreakitem
      \textit{\href{http://www.davidschaich.net/talks/SCGT12Mini.pdf}{Lattice Strong Dynamics for the LHC}} \\
      Conformality in Strong Coupling Gauge Theories at LHC and Lattice, \\ Kobayashi--Maskawa Institute, Nagoya University, 20 March 2012
    \pagebreakitem
      \textit{\href{http://www.davidschaich.net/talks/1202WW.pdf}{Lattice Strong Dynamics for the LHC: WW Scattering Parameters via Pseudoscalar Phase Shifts}} \\
      University of Colorado, 9 February 2012
    \pagebreakitem
      \textit{\href{http://www.davidschaich.net/talks/Lattice11.pdf}{$S$ parameter and parity doubling below the conformal window}} \\
      29th International Symposium on Lattice Field Theory, Lake Tahoe, CA, 12 July 2011
    \pagebreakitem
      \textit{\href{http://www.davidschaich.net/talks/defense.pdf}{Measuring the $S$ Parameter on the Lattice}} \\
      Boston University, 12 May 2011
    \pagebreakitem
      \textit{Monte Carlo Renormalization Group} \\
      MIT Lattice Club, 30 March 2011
    \pagebreakitem
      \textit{\href{http://www.davidschaich.net/talks/BUseminar.pdf}{Exploring the Origin of Mass with High-Performance Computing}} \\
      Boston University, 10 December 2010
    \pagebreakitem
      \textit{\href{http://www.davidschaich.net/talks/1010MIT.pdf}{Lattice Strong Dynamics for Electroweak Symmetry Breaking}} \\
      MIT Lattice Club, 20 October 2010
    \pagebreakitem
      \textit{\href{http://www.davidschaich.net/talks/Lattice10.pdf}{Flavor dependence of the $S$ parameter in $SU(3)$ gauge theory}} \\
      XXVIII International Symposium on Lattice Field Theory, Villasimius, Italy, 17 June 2010
    \pagebreakitem
      \textit{\href{http://www.davidschaich.net/talks/EWSB_lattice.pdf}{Exploring Electroweak Symmetry Breaking on the Lattice}} \\
      Boston University, 13 October 2009
    \pagebreakitem
      \textit{\href{http://www.davidschaich.net/talks/TC_LHC.pdf}{Technicolor at the LHC}} \\
      Boston University LHC Physics Symposium, 30 April 2009
    \pagebreakitem
      \textit{\href{http://www.davidschaich.net/talks/thesisDefense.pdf}{Lattice Simulations of Nonperturbative Quantum Field Theories}} \\
      Amherst College, 2 May 2006
    \pagebreakitem
      \textit{\href{http://www.davidschaich.net/talks/thesisTalk.pdf}{Life on the Lattice: Markov Chain Monte Carlo and all that}} \\
      Amherst College, 29 November 2005
    \pagebreakitem
      \textit{\href{http://www.davidschaich.net/talks/topMass.pdf}{Top Quark Physics at the LHC}} \\
      Five-College Physics Symposium, University of Massachusetts, 1 October 2005 \\
% ------------------------------------------------------------------
%
%
%
% ------------------------------------------------------------------
% TODO: Page break hack
%\newpage \hspace{-22 pt}{\large \bfseries Posters} \vspace{-8 pt}
\vspace{18 pt} \hspace{-22 pt}{\large \bfseries Posters} \vspace{-8 pt}
    \pagebreakitem
      \textit{\href{http://www.davidschaich.net/talks/SCGTposter.pdf}{Finite-temperature study of eight-flavor SU(3) gauge theory}} \\
      Origin of Mass and Strong Coupling Gauge Theories, \\ Kobayashi--Maskawa Institute, Nagoya University, 3 March 2015
    \pagebreakitem
      \textit{\href{http://www.davidschaich.net/talks/XQCD14.pdf}{Extremely supersymmetric lattice gauge theory}} \\
      eXtreme QCD Workshop on QCD under extreme conditions, Stony Brook University, 20 June 2014
    \pagebreakitem
      \textit{\href{http://www.davidschaich.net/talks/PPCM14.pdf}{Numerical Simulations of $\mathcal N = 4$ Supersymmetric Yang--Mills}} \\
      Field Theoretic Computer Simulations for Particle Physics and Condensed Matter, Boston U., 8 May 2014
    \pagebreakitem
      \textit{\href{http://www.davidschaich.net/talks/SITposter.pdf}{Exploring the Origin of Mass with High-Performance Computing}} \\
      National Science Foundation EAPSI Project Exhibition, 19 August 2011
    \pagebreakitem
      \textit{\href{http://www.davidschaich.net/talks/IGERT.pdf}{Lattice Strong Dynamics: Using high-performance computing to explore the mystery of mass}} \\
      National Science Foundation IGERT Project Meeting, Washington DC, 24--25 May 2010
    \pagebreakitem
      \textit{\href{http://www.davidschaich.net/talks/BUsymposium.pdf}{Lattice~Strong~Dynamics:~Using~high-performance~computing~to~explore~electroweak~symmetry~breaking}} \\
      Boston University Science and Engineering Research Symposium, 30 March 2010
    \pagebreakitem
      \textit{\href{http://www.davidschaich.net/talks/AAPT07.pdf}{Interdisciplinary Cluster Computing at a Liberal Arts College}} \\
      AAPT Topical Conference on Computational Physics for Upper Level Courses, Davidson Coll., 27-28 July 2007
  \end{revnumerate}
\end{spacelistout}
% ------------------------------------------------------------------

\end{document}
% ------------------------------------------------------------------
